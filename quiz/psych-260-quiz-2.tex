\documentclass[answers]{exam}
\usepackage{graphicx}
\usepackage{wrapfig}
\usepackage[utf8]{inputenc}

\title{PSYCH 260-BBH 203 Quiz 2}
\author{}
\date{October 5, 2015}

\pagestyle{headandfoot}
\firstpageheader{PSY 260-BBH 203}{Section 002}{Quiz 2}
\runningheader{PSY 260-BBH 203}{Section 002}{Quiz 2}
\firstpagefooter{}{Page \thepage\ of \numpages}{}
\runningfooter{}{Page \thepage\ of \numpages}{}

\begin{document}
\maketitle

\begin{center}
  \fbox{\fbox{\parbox{5.5in}{\centering
        Answer the questions using the Scantron form.}}}
\end{center}
\vspace{0.1in}
\makebox[\textwidth]{Name:\enspace\hrulefill}

\newpage

\section{Main}

\begin{questions}

\question The ventral \fillin area contains neurons that release \fillin.
\begin{choices}
\choice striatum; oxytocin
\correctchoice tegmental; dopamine 
\choice tectal; glycine
\choice substantia nigra; serotonin
\end{choices}

\question An SSRI has what effect on serotonin?
\begin{choices}
\choice Reduces it temporarily.
\choice Accelerates its reuptake.
\correctchoice Increases levels at the synapse.
\choice Causes it to binds to ionotropic receptors.
\end{choices}

\question Hormones are NOT released from which brain structure?
\begin{choices}
\correctchoice Thalamus.
\choice Hypothalamus.
\choice Pineal gland.
\choice Pituitary.
\end{choices}

\question \fillin is the most commonly released neurotransmitter in the CNS. It typically binds to a/an \fillin receptor and has a/an \fillin effect.
\begin{choices}
\choice GABA; metabotropic; excitatory
\choice ACh; metabotropic; inhibitory.
\correctchoice Glutamate; ionotropic; excitatory.
\choice Dopamine; metabotropic; modulatory.
\end{choices}

\question Which of these hormones is released by the anterior pituitary?
\begin{choices}
\choice Oxytocin.
\choice Cortisol.
\choice Vasopressin.
\correctchoice Adrenocorticotropic hormone (ACTH).
\end{choices}

\question Which of these is the primary output neurotransmitter of the CNS?
\begin{choices}
\choice Dopamine.
\choice Norepinephrine.
\correctchoice Acetylcholine. 
\choice Melatonin.
\end{choices}

\question \fillin receptors contain both chemical (ligand) binding sites and an ion channel.
\begin{choices}
\choice Acetylcholine
\choice Serotonin
\choice Glutamate
\correctchoice Ionotropic
\end{choices}

\newpage

\question The inward flow of \fillin across the neural membrane creates an \fillin.
\begin{choices}
\correctchoice Cl-; IPSP
\choice K+; IPSP
\choice Glutamate; EPSP
\choice GABA; IPSP
\end{choices}

\question The histamine neurotransmitter is released from the \fillin and is involved in \fillin.
\begin{choices}
\choice Pineal Gland; Sleep
\choice Raphe Nuclei; Mood
\choice Hypothalamus; Immune Response
\choice Pons; Arousal
\end{choices}

\question Visual information from the lateral geniculate nucleus in the thalamus projects to this part of the brain.
\begin{choices}
\choice Temporal lobe
\correctchoice Occipital lobe
\choice Raphe nucleus
\choice Hippocampus
\end{choices}

\section{Bonus}

\question Which neurotransmitter is related to the umami (savory) taste sensation?
\begin{choices}
\correctchoice Glutamate
\choice GABA
\choice Glycine
\choice Aspartate
\end{choices}

\question You are examining a/an \fillin synapse. Based on where it connects on the cell body, you guess that it is \fillin and involves the release of \fillin.
\begin{choices}
\choice dendrodendritic; modulatory; Ca++
\correctchoice axosomatic; inhibitory; GABA
\choice axoaxonic; excitatory; adenosine
\choice axodendritic; inhibitory; glutamate
\end{choices}

\end{questions}
\end{document}